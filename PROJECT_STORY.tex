\documentclass[11pt,a4paper]{article}
\usepackage[utf8]{inputenc}
\usepackage[T1]{fontenc}
\usepackage{amsmath,amssymb,amsfonts}
\usepackage{graphicx}
\usepackage{booktabs}
\usepackage{array}
\usepackage{geometry}
\usepackage{hyperref}
\usepackage{xcolor}
\usepackage{listings}
\usepackage{float}

\geometry{margin=1in}
\hypersetup{colorlinks=true,linkcolor=blue,urlcolor=blue}

\title{\textbf{EquineSync: Early Detection for Healthier Horses}}
\author{Tableau Hackathon Submission}
\date{}

\begin{document}

\maketitle

\section{Inspiration}

This project was born from personal heartbreak. I purchased a horse that appeared healthy during the sale, but within weeks, I discovered he was lame. The lameness wasn't visible to the untrained eye during a brief viewing—it only became apparent through careful observation over time.

\textbf{I couldn't return him. I couldn't resell him. But I couldn't abandon him either.}

Caring for a lame horse is expensive and emotionally draining. Veterinary bills, special shoeing, supplements, and rehabilitation add up quickly. But more than the cost, I felt deceived—and I knew I wasn't alone. Thousands of horses are sold each year with undetected lameness, leading to:

\begin{itemize}
    \item Financial hardship for unsuspecting buyers
    \item Horses being abandoned or sent to slaughter when owners can't cope
    \item Preventable suffering that early detection could have avoided
\end{itemize}

\textbf{I decided to build a solution.} If I win any prize money from this hackathon, every penny will go toward giving my lame horse a better life—and developing this technology to help other horses get diagnosed early, before it's too late.

\section{What It Does}

\textbf{EquineSync} is an AI-powered analytics platform that monitors four-legged animals (primarily horses) for early signs of lameness and health issues through:

\begin{enumerate}
    \item \textbf{Gait Symmetry Analysis} — Detecting subtle asymmetries across all four legs with real-time visualization
    \item \textbf{Heart Rate Variability (HRV) Monitoring} — Assessing stress and cardiac health using ESC/NASPE standards
    \item \textbf{Individual Leg Health Scoring} — 0--100 health scores for each leg with predictive deductions
    \item \textbf{Horse-Rider Emotional Bond Tracking} — Monitoring emotional synchronization and stress impact on partnership
    \item \textbf{Real-Time Leg Detection} — Auto-identifying which sensor is on which leg
    \item \textbf{Session Result Reports} — Period-by-period behavioral analysis with downloadable comprehensive reports
    \item \textbf{Alert System} — Flagging potential issues before they become serious
    \item \textbf{Live Event Logging} — Session progress tracking with timestamped milestone events
\end{enumerate}

\subsection{Advanced Dashboard Features}

The EquineSync real-time monitoring dashboard provides:

\begin{itemize}
    \item \textbf{Live Visualization}: Real-time charts using Chart.js displaying:
    \begin{itemize}
        \item Symmetry evolution (overall, front, hind) over session duration
        \item 4-leg amplitude tracking with color-coded lines
        \item HRV trend analysis (SDNN over time)
        \item Horse-rider bond score evolution
    \end{itemize}

    \item \textbf{Session Progress Log}: Live event feed showing:
    \begin{itemize}
        \item System initialization and milestone events
        \item Alert triggers with timestamps
        \item Health score changes and degradation warnings
        \item Critical events (e.g., lameness onset detection)
    \end{itemize}

    \item \textbf{Period-by-Period Analysis}: Automatic breakdown into 30-second periods showing:
    \begin{itemize}
        \item Average symmetry and bond scores per period
        \item Behavioral observations (baseline, onset, compensation)
        \item Color-coded status indicators (green/orange/red)
        \item Comparative analysis across time periods
    \end{itemize}

    \item \textbf{Downloadable Reports}: Comprehensive text-based clinical reports including:
    \begin{itemize}
        \item Minute-by-minute behavioral analysis
        \item Statistical summaries (min/max/average metrics)
        \item Alert history and timeline
        \item Clinical recommendations with urgency levels
        \item Veterinary action items and follow-up protocols
    \end{itemize}
\end{itemize}

\section{How I Built It}

\subsection{Architecture}

\begin{verbatim}
+--------------+   +----------------+   +------------+   +-----------+
| IMU Sensors  |-->| Confluent Cloud|-->|  Vertex AI |-->|  Tableau  |
|  (4 legs)    |   |  Kafka Streams |   |  ML Models |   | Dashboard |
| 100Hz stream |   |                |   |            |   |           |
+--------------+   |  Topics:       |   +------------+   +-----------+
| Heart Rate   |-->|  - sensor-raw  |-->| Gait Anal. |-->|   Slack   |
|   Monitor    |   |  - gait-data   |   | HRV Anal.  |   |  Alerts   |
|              |   |  - hrv-metrics |   | Anomaly    |   |           |
+--------------+   |  - alerts      |   +------------+   +-----------+
                   +----------------+

Real-time streaming: Sensors -> Kafka -> AI -> Insights
\end{verbatim}

\textbf{Key Components:}
\begin{itemize}
    \item \textbf{Confluent Cloud}: Apache Kafka for real-time data streaming (4 sensors $\times$ 100Hz = 400 data points/sec)
    \item \textbf{Vertex AI}: Google Cloud ML for streaming inference on gait symmetry and HRV patterns
    \item \textbf{Tableau Cloud}: Real-time visualization of health metrics and trends
    \item \textbf{Slack Integration}: Instant alerts when lameness or stress detected
\end{itemize}

\subsection{Real-Time Streaming with Confluent}

\textbf{Why Confluent + Google Cloud?}

Traditional batch processing would miss critical moments—lameness develops over seconds, not hours. By streaming sensor data through Confluent Kafka and processing with Vertex AI in real-time, we detect abnormalities \textbf{within milliseconds} of occurrence.

\vspace{0.5em}
\noindent\textbf{Kafka Topics Architecture:}

\begin{enumerate}
    \item \texttt{sensor-data-raw} (400 msgs/sec)
    \begin{itemize}
        \item Raw IMU readings from 4 legs (accelerometer + gyroscope)
        \item Heart rate monitor R--R intervals
        \item Timestamped, partitioned by sensor ID
    \end{itemize}

    \item \texttt{gait-analysis} (processed stream)
    \begin{itemize}
        \item Stride frequency, symmetry scores, leg health metrics
        \item Computed in real-time via Vertex AI inference
        \item Triggers alerts when thresholds breached
    \end{itemize}

    \item \texttt{hrv-metrics} (1 msg/sec)
    \begin{itemize}
        \item SDNN, RMSSD, pNN50 calculations
        \item Stress level indicators
        \item Emotional state tracking (horse-rider bond)
    \end{itemize}

    \item \texttt{alerts} (event-driven)
    \begin{itemize}
        \item Asymmetry alerts (symmetry $< 60$ for $>3$ readings)
        \item Impact alerts (peak acceleration $> 2.5\times$ baseline)
        \item HRV alerts (stress indicators in danger zone)
    \end{itemize}
\end{enumerate}

\vspace{0.5em}
\noindent\textbf{Performance Benefits:}
\begin{itemize}
    \item \textbf{Latency}: $<$50ms from sensor reading to alert
    \item \textbf{Throughput}: Handles 400 data points/sec with room to scale
    \item \textbf{Reliability}: Kafka replication ensures no data loss
    \item \textbf{Scalability}: Can monitor entire stables (100+ horses) simultaneously
\end{itemize}

\subsection{Mathematical Foundation}

\subsubsection{Nomenclature}

\begin{table}[H]
\centering
\begin{tabular}{@{}lll@{}}
\toprule
\textbf{Symbol} & \textbf{Description} & \textbf{Units} \\
\midrule
$RR_i$ & $i$-th R--R interval & ms \\
$A_{FL}$ & Normalized amplitude (front left) & \% \\
$k_{front}$ & Front symmetry scaling factor & dimensionless \\
$\sigma_{stride}$ & Stride interval standard deviation & s \\
$\sigma_{normal}$ & Normal stride variability baseline & s \\
$\sigma_{baseline}$ & Horse's baseline stride variability & s \\
$f_{expected}$ & Expected stride frequency for gait & Hz \\
$f_{measured}$ & Observed stride frequency & Hz \\
$a_{peak}$ & Peak vertical acceleration & $g$ \\
$a_{baseline}$ & Baseline acceleration reference & $g$ \\
$a_{i}$ & Acceleration vector for stride segment $i$ & $g$ \\
$a_{healthy,i}$ & Baseline healthy acceleration template & $g$ \\
$C_{freq}$ & Frequency-based confidence score & \% \\
$S_{total}$ & Overall symmetry score & 0--100 \\
$S_{pair}$ & Leg pair symmetry score & 0--100 \\
\bottomrule
\end{tabular}
\end{table}

\subsubsection{Heart Rate Variability (HRV) Analysis}

Raw R-R intervals are filtered to remove artifacts using physiological bounds:

\begin{equation}
300\text{ ms} \leq RR_i \leq 2000\text{ ms}
\end{equation}

\begin{equation}
|RR_i - \text{median}(RR)| \leq 0.20 \times \text{median}(RR)
\end{equation}

\textbf{Key HRV Metrics:}

\vspace{0.5em}
\noindent\textbf{SDNN} (Standard Deviation of NN intervals) — measures overall HRV:
\begin{equation}
SDNN = \sqrt{\frac{1}{N-1}\sum_{i=1}^{N}(RR_i - \overline{RR})^2}
\end{equation}

\noindent\textbf{RMSSD} (Root Mean Square of Successive Differences) — measures parasympathetic activity:
\begin{equation}
RMSSD = \sqrt{\frac{1}{N-1}\sum_{i=1}^{N-1}(RR_{i+1} - RR_i)^2}
\end{equation}

\noindent\textbf{pNN50} (Percentage of successive intervals differing by $>$50ms):
\begin{equation}
pNN50 = \frac{\text{count}(|RR_{i+1} - RR_i| > 50\text{ ms})}{N-1} \times 100\%
\end{equation}

\begin{table}[H]
\centering
\caption{HRV Metric Interpretation Thresholds}
\begin{tabular}{@{}lccc@{}}
\toprule
\textbf{Metric} & \textbf{Good} & \textbf{Warning} & \textbf{Alert} \\
\midrule
SDNN & $>50$ ms & $30$--$50$ ms & $<30$ ms \\
RMSSD & $>40$ ms & $20$--$40$ ms & $<20$ ms \\
pNN50 & $>3\%$ & $1$--$3\%$ & $<1\%$ \\
\bottomrule
\end{tabular}\\[0.3em]
\footnotesize\textit{Thresholds adapted from equine HRV studies [Ref 2] and scaled from human ESC/NASPE guidelines [Ref 1]. Values are indicative and should be calibrated per individual horse.}
\end{table}

\subsubsection{Gait Classification}

IMU sensors scaled at $\pm16g$ acceleration and $\pm2000°/s$ gyroscope measure stride patterns:

\begin{table}[H]
\centering
\caption{Gait Classification Thresholds}
\begin{tabular}{@{}lll@{}}
\toprule
\textbf{Gait} & \textbf{Stride Frequency} & \textbf{Pattern} \\
\midrule
Stand & $<0.3$ Hz & Minimal movement \\
Walk & $0.3$--$1.0$ Hz & Regular, low amplitude \\
Trot & $1.0$--$1.8$ Hz & Two-beat diagonal \\
Canter & $1.8$--$2.5$ Hz & Three-beat asymmetric \\
Gallop & $>2.5$ Hz & Four-beat high amplitude \\
\bottomrule
\end{tabular}
\end{table}

\subsubsection{Symmetry Analysis}

For each leg pair, symmetry is calculated as follows:

\vspace{0.5em}
\noindent\textbf{Front Pair Symmetry:}
\begin{equation}
S_{front} = 100 - |A_{FL} - A_{FR}| \times k_{front} \quad [\text{dimensionless}]
\end{equation}

\noindent\textbf{Hind Pair Symmetry:}
\begin{equation}
S_{hind} = 100 - |A_{BL} - A_{BR}| \times k_{hind} \quad [\text{dimensionless}]
\end{equation}

\noindent\textbf{Diagonal Symmetry:}
\begin{equation}
S_{diag} = 100 - \frac{|A_{FL} - A_{BR}| + |A_{FR} - A_{BL}|}{2} \times k_{diag} \quad [\text{dimensionless}]
\end{equation}

\vspace{0.5em}
\noindent Where:
\begin{itemize}
    \item $A_{FL}, A_{FR}, A_{BL}, A_{BR}$ are normalized stride amplitude ratios (peak vertical acceleration divided by baseline, expressed as \%)
    \item Scaling factors: $k_{front}=25$, $k_{hind}=25$, $k_{diag}=30$ (empirically tuned to map 0--4\% asymmetry to a 0--100 score)
\end{itemize}

\noindent\textbf{Overall Symmetry Score:}
\begin{equation}
S_{total} = w_1 \cdot S_{front} + w_2 \cdot S_{hind} + w_3 \cdot S_{diag}
\end{equation}

Where $w_1 = 0.35$, $w_2 = 0.35$, $w_3 = 0.30$.

\begin{table}[H]
\centering
\caption{Symmetry Score Interpretation}
\begin{tabular}{@{}lll@{}}
\toprule
\textbf{Score} & \textbf{Status} & \textbf{Action} \\
\midrule
$\geq 80$ & Good & Continue monitoring \\
$60$--$79$ & Warning & Increase observation \\
$< 60$ & Alert & Veterinary consultation recommended \\
\bottomrule
\end{tabular}
\end{table}

\subsubsection{Automatic Leg Detection}

The system auto-identifies which sensor is on which leg using a calibration walk.

\vspace{0.5em}
\noindent\textbf{Motion Activation Threshold:}
\begin{equation}
E_{motion} = \int_{t}^{t+2s} (a_x^2 + a_y^2 + a_z^2) \, dt > 5.0 \text{ } g^2 \cdot s
\end{equation}

\noindent\textbf{Front/Hind Classification} (via acceleration axis ratio):
\begin{equation}
R_{FH} = \frac{\sigma(a_{vertical})}{\sigma(a_{horizontal})}
\end{equation}

Classification rules:
\begin{itemize}
    \item Front legs: $R_{FH} > 1.2$
    \item Hind legs: $R_{FH} < 0.8$
\end{itemize}

\noindent\textbf{Left/Right Classification} (via cross-correlation):
\begin{equation}
\rho_{LR} = \frac{\text{cov}(a_{lateral,1}, a_{lateral,2})}{\sigma_1 \cdot \sigma_2}
\end{equation}

\noindent\textbf{Overall Confidence Score:}
\begin{equation}
C_{total} = 0.40 \cdot C_{freq} + 0.30 \cdot C_{FH} + 0.20 \cdot C_{LR} + 0.10 \cdot C_{diag}
\end{equation}

\vspace{0.5em}
\noindent Weights determined via logistic regression on labeled training data ($n=120$ strides), where:
\begin{itemize}
    \item $C_{freq}$: Gait frequency confidence (most discriminative)
    \item $C_{FH}$: Front/hind classification confidence
    \item $C_{LR}$: Left/right classification confidence
    \item $C_{diag}$: Diagonal correlation confidence
\end{itemize}

\subsubsection{Leg Health Scoring}

\begin{samepage}
Each leg receives a health score based on multiple factors:

\begin{equation}
Score_{leg} = 100 - D_{variability} - D_{frequency} - D_{deviation}
\end{equation}

Where deductions are explicitly calculated as:

\begin{equation}
D_{variability} = 20 \times \frac{\sigma_{stride}}{\sigma_{baseline}} \quad \text{(capped at 40)}
\end{equation}

\begin{equation}
D_{frequency} = 15 \times \frac{|f_{measured} - f_{expected}|}{f_{expected}} \quad \text{(capped at 30)}
\end{equation}

\begin{equation}
D_{deviation} = 10 \times \frac{\sum |a_{i} - a_{healthy,i}|}{\sum a_{healthy,i}} \quad \text{(capped at 30)}
\end{equation}

\vspace{0.5em}
\noindent Where:
\begin{itemize}
    \item $\sigma_{stride}$: Standard deviation of stride intervals over 1 minute
    \item $\sigma_{baseline}$: Horse's baseline stride variability
    \item $f_{measured}$: Observed stride frequency
    \item $f_{expected}$: Expected frequency for current gait (from Table 2)
    \item $a_{i}$: Acceleration vector for stride segment $i$
    \item $a_{healthy,i}$: Baseline healthy acceleration template
\end{itemize}
\end{samepage}

\subsubsection{Alert Generation}

Alerts are triggered under the following conditions:

\vspace{0.5em}
\noindent\textbf{Asymmetry Alert:}
\begin{equation}
|S_{pair}| < 60 \text{ for } > 3 \text{ consecutive readings}
\end{equation}

\noindent\textbf{Impact Alert:}
\begin{equation}
a_{peak} > 2.5 \times a_{baseline}
\end{equation}

\noindent\textbf{Irregularity Alert:}
\begin{equation}
\sigma_{stride} > 1.5 \times \sigma_{normal}
\end{equation}

\subsubsection{Horse-Rider Emotional Bond Analysis}

The horse-rider bond score quantifies the emotional connection and partnership quality:

\begin{equation}
Bond_{score} = Bond_{base} + \alpha \cdot (SDNN - SDNN_{ref}) - \beta \cdot t_{stress}
\end{equation}

Where:
\begin{itemize}
    \item $Bond_{base}$: Initial bond score (typically 85--95 for established partnerships)
    \item $\alpha = 0.3$: HRV adjustment factor
    \item $SDNN_{ref} = 40$ ms: Reference SDNN value
    \item $\beta = 0.4$: Stress duration penalty factor (points per second)
    \item $t_{stress}$: Time elapsed since stress onset (e.g., lameness detection)
\end{itemize}

\vspace{0.5em}
\noindent The bond score is clamped to $[0, 100]$ and interpreted as:

\begin{table}[H]
\centering
\caption{Horse-Rider Bond Score Interpretation}
\begin{tabular}{@{}ll@{}}
\toprule
\textbf{Score Range} & \textbf{Interpretation} \\
\midrule
$80$--$100$ & Excellent Partnership \\
$65$--$79$ & Strong Connection \\
$50$--$64$ & Moderate Bond \\
$< 50$ & Stress Affecting Bond \\
\bottomrule
\end{tabular}
\end{table}

\noindent\textbf{Clinical Significance:} Bond degradation indicates that pain or discomfort is affecting the horse's responsiveness to rider cues, impacting training effectiveness and partnership quality.

\subsubsection{Kalman Filter for Label Stability}

To maintain stable leg identification over time, a Kalman filter is applied:

\begin{equation}
\hat{x}_k = \hat{x}_{k-1} + K_k(z_k - \hat{x}_{k-1})
\end{equation}

Where:
\begin{itemize}
    \item $\hat{x}_k$ = estimated state at time $k$
    \item $K_k$ = Kalman gain (adaptive threshold)
    \item $z_k$ = measurement at time $k$
\end{itemize}

\section{Challenges I Faced}

\subsection{Signal Noise}
Real-world sensor data is messy. Horse movement creates vibrations, and sensors can shift during exercise. The Kalman filtering approach (Equation 21) was essential for maintaining stable readings.

\subsection{Individual Variation}
Every horse moves differently. A Thoroughbred's gait differs significantly from a Draft horse's. The system required adaptive baselines that learn each horse's ``normal'' movement patterns over time.

\subsection{Balancing Sensitivity}
\begin{itemize}
    \item Too sensitive $\rightarrow$ false alarms causing unnecessary worry
    \item Too lenient $\rightarrow$ missed detections defeating the purpose
\end{itemize}
Thresholds were tuned based on veterinary literature and ESC/NASPE HRV standards.

\subsection{Making It Accessible}
Complex biomechanical data needs to be understandable for trainers without medical degrees. Tableau's visualization capabilities were essential for translating raw numbers into actionable insights.

\section{What I Learned}

\begin{itemize}
    \item \textbf{Real-time data streaming} — Building scalable event-driven architectures with Confluent Kafka
    \item \textbf{Google Cloud AI} — Deploying Vertex AI models for streaming ML inference at scale
    \item \textbf{Equine biomechanics} — How horses distribute weight, compensate for pain, and how subtle lameness manifests
    \item \textbf{Signal processing} — FFT for frequency analysis, cross-correlation for leg pairing, Kalman filtering for stability
    \item \textbf{HRV science} — The parasympathetic nervous system's role in stress detection and emotional bond assessment
    \item \textbf{Stream processing patterns} — Handling 400+ messages/sec with low latency and high reliability
    \item \textbf{Tableau Developer Platform} — APIs, embedding, and real-time data integration
    \item \textbf{Data visualization} — Implementing real-time Chart.js visualizations for biomechanical data
    \item \textbf{Authentic research data integration} — Processing and streaming the ``Horsing Around'' dataset (4TU ResearchData, CC0 license) containing genuine equine IMU data
    \item \textbf{Clinical report generation} — Creating actionable, veterinary-grade reports with period-by-period analysis
\end{itemize}

\section{Future Improvements}

Given more time, I would:

\begin{enumerate}
    \item \textbf{Add computer vision} — Camera-based gait analysis to complement IMU sensors
    \item \textbf{Build mobile app} — On-field assessments during horse sales
    \item \textbf{Create marketplace integration} — Verified health certificates for horse sales
    \item \textbf{Expand to other animals} — Dogs, livestock, zoo animals
\end{enumerate}

\section{My Promise}

\textbf{If this project wins any prize money, 100\% will be dedicated to:}

\begin{enumerate}
    \item Providing the best possible care for my lame horse
    \item Developing this technology further to help other horses
    \item Working with rescue organizations to screen horses before adoption
\end{enumerate}

No horse should suffer because their lameness wasn't caught early. No buyer should face the heartbreak I experienced. Technology can help—and that's what EquineSync is all about.

\section{References}

\begin{enumerate}
    \item Task Force of the European Society of Cardiology and the North American Society of Pacing and Electrophysiology. ``Heart rate variability: standards of measurement, physiological interpretation, and clinical use.'' \textit{Circulation} 93.5 (1996): 1043--1065.
    \item Equine heart rate variability research literature
    \item Gait biomechanics and lameness detection studies
\end{enumerate}

\vspace{1em}
\hrule
\vspace{0.5em}

\noindent\textbf{Built With:}

\textbf{Data Streaming \& Processing:}
\begin{itemize}
    \item Confluent Cloud (Apache Kafka)
    \item Stream processing at 400 msgs/sec
    \item Flask API server for real-time data endpoints
\end{itemize}

\textbf{AI \& Machine Learning:}
\begin{itemize}
    \item Google Vertex AI / Gemini
    \item Python (NumPy, SciPy, Scikit-learn, Pandas)
    \item Real-time ML inference
\end{itemize}

\textbf{Cloud Platform:}
\begin{itemize}
    \item Google Cloud Platform
    \item Tableau Cloud
    \item Cloud Run, Cloud Storage, BigQuery
\end{itemize}

\textbf{Frontend \& Visualization:}
\begin{itemize}
    \item React.js (Main dashboard UI)
    \item Chart.js (Real-time data visualization)
    \item HTML5/CSS3 (Advanced monitoring interface)
    \item Vite (Build tooling and optimization)
\end{itemize}

\textbf{Data Sources:}
\begin{itemize}
    \item Horsing Around Dataset (4TU ResearchData, CC0 License)
    \item 10.96 GB authentic equine IMU data
    \item 148 CSV files with 100Hz sampling rate
\end{itemize}

\textbf{Hardware:}
\begin{itemize}
    \item IMU Sensors ($\pm$16g accelerometer, $\pm$2000°/s gyroscope)
    \item Heart Rate Monitors (R--R interval detection)
\end{itemize}

\textbf{Integration:}
\begin{itemize}
    \item Slack API (alerts)
    \item Confluent Connectors (GCP integration)
    \item REST APIs
\end{itemize}

\vspace{0.5em}
\textit{Built with Confluent Kafka, Vertex AI, Tableau Cloud, React, Chart.js, Python, authentic research data, and a deep love for horses.}

\end{document}
